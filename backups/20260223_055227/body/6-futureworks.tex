透過本研究對不同語音基礎模型在深偽偵測上的效能分析與討論,我們不僅看見了目前主流架構在跨域泛化上的限制,也為後續研究指引了數個具潛力的發展方向。未來的工作不僅限於提升單一資料集的準確率,更應著眼於提升模型的實用性與強健性,我們提出以下四點未來展望:

\section{結合連續聲學特徵與離散語意的多模態融合}
基於第五章的結論,強制特徵離散化會導致用以辨別真偽的微小瑕疵(如高頻假影或壓縮損失)被當作雜訊過濾;然而,W2V-AASIST 所代表的連續聲學特徵模型雖然精準,卻可能在面對具有極端變異(如 MLAAD 資料集)的場景時,反而受到非本身語音特徵的雜訊誤導。未來研究可嘗試構建「雙流架構 (Two-stream Architecture)」,同時輸入未經壓縮的連續波形特徵與具有高階語境意義的離散 Token,將兩者的優勢結合,讓模型既能兼顧局部的聲學瑕疵,又能捕捉全域的序列與語意異常。

\section{參數效率微調 (PEFT) 技術的進階應用}
雖實驗證實 SpeechPrompt v2 受限於離散化特徵而存在效能天花板,但其展現出的「5 分鐘對抗 36 小時」訓練效率優勢(約高達四百倍的加速)在實際工業部署中深具吸引力。未來可延續「參數效率微調 (Parameter-Efficient Fine-Tuning)」的思維,放棄純粹的 Prompt Tuning,改以在連續特徵網路中插入輕量級的 Adapter 模組,或採用 Low-Rank Adaptation (LoRA) 等技術。期盼能在不全模型更新的前提下,以極低的參數量訓練出具備極佳跨域防禦力的輕量化模型。

\section{增強針對真實場景 (In-the-Wild) 變異的強健性}
目前的模型在面對 In-The-Wild (ITW) 與 ASV21 DeepFake 測試集時,無論是連續或離散特徵架構皆出現了顯著的防禦力衰退。這顯示未知環境噪音、通道失配 (Channel Mismatch) 以及強烈的有損壓縮對於深偽特徵的掩蔽效應十分嚴重。未來的系統應整合專門針對語音還原與雜訊抑制的前處理網路,或在訓練階段引入對抗性生成訓練 (Adversarial Training),動態生成足以混淆模型的邊界樣本,以提升真實場景下的通用性。

\section{超越簡單擴容的前端特徵多層次聚合後端}
正如 MFA-Conformer 實驗中所揭示,單純增加預訓練前端特徵抽取器的模型規模(從 Small 擴展至 Large),並無法有效突破跨領域偵測的效能飽和點。未來的研究重心應從「堆疊更龐大的前端架構」轉移至「設計更強大的後端特徵融合網路」。如本研究初步驗證的 AASIST 後端,利用圖神經網路 (Graph Neural Network) 或是自注意力機制探索來自不同特徵層(空間、頻率、語意)間的複雜關聯,才能更全面地利用預訓練模型在各層級所學到的豐富表徵。