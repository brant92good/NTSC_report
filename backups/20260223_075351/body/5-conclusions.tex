透過前述章節的各項實驗與量化評估,本研究針對「基於語音基礎模型之深偽語音偵測」得到了數項重要的觀察與結論,並針對其中不尋常之實驗現象(如 MLAAD 測試集之極端表現)進行了進一步的推論與探討。

\section{單一特徵域的過擬合與顯著的跨域效能衰退}
儘管我們觀察到 W2V-AASIST 與 MFA-Conformer 在同領域(In-domain)的基準測試(如 ASV19 LA)中皆能展現優異甚至接近完美的偵測準確率,但當測試場景轉移至存在未知編碼變異、通道失配或強烈環境噪音的跨域資料集(如 ASV21 DF 或 InTheWild)時,所有模型的效能皆不可避免地出現了極其顯著的衰退(EER 普遍呈數十倍攀升)。這項結果證明,目前無論是連續或離散特徵型態的語音基礎模型,在進行單一資料集訓練時,仍極易對該資料集特定的錄音室聲學環境或深偽演算法產生嚴重的過擬合(Overfitting)。模型並未真正學到放諸四海皆準的「偽造本質」,而是記住了特徵空間中的捷徑與分佈捷徑,導致泛化能力(Generalization Ability)在面對高複雜度的現代偽造攻擊時依然十分堪憂。

\section{特徵離散化對泛化能力之限制}
前述的跨域效能衰退中,又以特徵離散化模型最為嚴重。實驗結果表明,前端聲學特徵的「離散化」過程(如 SpeechPrompt v2 中將 HuBERT 連續特徵轉換為主成分中心點單元)是限制模型應付未知跨領域攻擊的核心瓶頸。儘管在同領域測試中,離散特徵依然能藉由強大的後端語言模型達到一定水準的辨識率,或是透過極少量的 Prompt 參數即可初步激發分類能力,但當面臨未曾在訓練集中見過的壓縮波形或各式未知偽造演算法時,其表現遠遠落後於連續特徵架構。

我們推測,深偽音訊與真實語音之間的本質差異,往往存在於極其微小且難以察覺的連續聲學特徵變化或是高頻響應的扭曲中。然而,硬性量化到有限集合離散單元的過程,實質上等同於一個極端不可逆的低通濾波器或資料平滑操作,直接將這類用以區分真偽的關鍵微小瑕疵視為「雜訊」給抹除。因此,系統從源頭便喪失了建立穩健防禦決策所需的關鍵高頻與細節特徵。

\section{MLAAD 測試集的異常高準確率分析}
在跨領域測試中,我們觀察到一個極度反常的現象:在多數測試集中表現最差、幾乎不具備跨域能力(EER 極高)的 SpeechPrompt v2,卻在 MLAADv2 與 MLAADv8 上達到將近 100\% 的完美準確率;反而是在其他標準與真實場景測試集上最為穩健的連績特徵架構 W2V-AASIST,在此資料集上相對表現不佳(準確率約七至八成)。

針對此一反轉,我們認為並非 SpeechPrompt 具備某種未知的強大泛化性,而是反映了 MLAAD 資料集本身存在潛在的特性落差。具體而言:
\begin{itemize}
    \item \textbf{資料集潛在的捷徑 (Shortcut) 特徵}:MLAAD 的假音源或採樣方式可能帶有某種極為固定的背景噪音、特定編碼偽影或特殊的頻譜破壞,而這種單一破壞剛好對應到離散化模型中的特定單位群。使得模型無須學習「深偽本質」,只需退化成一個簡單的「特定離散單元或噪聲分類器」即可在一一對應的條件下完美分類。而恰好 SpeechPrompt 的語言模型架構本身便非常擅長捕捉這類離散序列的出現頻率。
    \item \textbf{連續特徵對未知極端變異的反效果}:相較之下,W2V-AASIST 依賴更細緻的連續特徵學習,當 MLAAD 中存在與 ASV19 或 InTheWild 截然不同的強烈聲學特徵變異(例如:不同空間的極端殘響或是極差的音質差異)時,這種過大的變異可能反而跨越了判別邊界,導致原本仰賴微小連續分佈特徵的分類網路發生混淆,從而降低準確率。
\end{itemize}

\section{前端多層次聚合與後端分類器設計之必要性}
在 MFA-Conformer 的次要實驗中,我們證實了僅有強大的前端特徵提取網路並不足以構成優秀的深偽偵測防禦系統。即使擴展了 Conformer 的模型規模(從 Small 擴增至 Large),依然無法跨越其在 ASV21 DF 上的效能天花板,表現呈現飽和狀態。然而,當我們嘗試將 Conformer 中所有特徵層 (All layer) 的資訊全數提取,並將後端替換為能進行複雜特徵融合的圖神經網路(AASIST 架構)時,效能產生了顯著躍升。這證明了「前端特徵的多層次聚合能力」與「後端的複雜分類決策」兩者在應對未知跨領域攻擊時缺一不可。

