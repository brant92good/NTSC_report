綜合第五章之結論,未來的語音深偽偵測系統若要提升對未知攻擊的泛化與防禦能力,不應在第一步就採取極度壓縮的離散化語意特徵輸入。相對地,盡量保留連續波形的聲學細節特徵,並設計能充分整合每一層表徵的特徵融合機制,才是突破目前跨網域深偽偵測瓶頸的發展主軸。沿此脈絡,我們不僅看見了目前主流架構在跨域泛化上的限制,也為後續研究指引了數個具潛力的發展方向。未來的工作不應僅限於提升單一資料集的準確率,更應著眼於提升模型的實用性與環境強健性,我們提出以下四點具體之未來展望:

\section{結合連續聲學特徵與離散語意的多模態融合}
基於第五章的結論,強制特徵離散化會導致用以辨別真偽的微小瑕疵(如高頻假影或壓縮損失)被當作雜訊過濾;然而,W2V-AASIST 所代表的連續聲學特徵模型雖然精準,卻可能在面對具有極端變異(如 MLAAD 資料集)的場景時,反而受到非本身語音特徵的雜訊誤導。未來研究可嘗試構建「雙流架構 (Two-stream Architecture)」,同時輸入未經壓縮的連續波形特徵與具有高階語境意義的離散 Token,將兩者的優勢結合,讓模型既能兼顧局部的聲學瑕疵,又能捕捉全域的序列與語意異常。

\section{探索基於連續特徵之語音語言模型}
本次實驗中,我們選用的 SpeechPrompt v2 是基於「離散單元 (Discrete Tokens)」作為輸入的語音語言模型。儘管此類模型在自然語言相關的語意推論上表現不俗,但在深偽偵測任務中,特徵的強制離散化卻成為流失防偽微瑕疵的致命傷。未來研究應轉向評估「直接接收連續聲學特徵 (Continuous Features)」的新一代語音語言模型。在保留連續空間資訊並避免量化損失的前提下,探討大語言模型架構是否能同時保有強大的泛化推理能力,以及具競爭力的深偽偵測效能,藉此根本性地彌補目前架構的弱點。

\section{增強針對真實場景 (In-the-Wild) 變異的強健性}
目前的模型在面對 In-The-Wild (ITW) 與 ASV21 DeepFake 測試集時,無論是連續或離散特徵架構皆出現了顯著的防禦力衰退。這顯示未知環境噪音、通道失配 (Channel Mismatch) 以及強烈的有損壓縮對於深偽特徵的掩蔽效應十分嚴重。未來的系統應整合專門針對語音還原與雜訊抑制的前處理網路,或在訓練階段引入對抗性生成訓練 (Adversarial Training),動態生成足以混淆模型的邊界樣本,以提升真實場景下的通用性。

\section{超越單一特徵的多重 SSL 前端聚合架構 (Multi-SSL Fusion)}
正如 MFA-Conformer 實驗中所揭示,單純增加單一預訓練前端(如 Conformer-Small 擴展至 Large)的模型規模,並無法有效突破跨領域偵測的效能飽和點。參考近期在未知領域防偽任務上的前沿發展趨勢,未來的研究重心應從「單一前端的堆疊與複雜後端設計」,轉移至「多重自監督學習 (Multi-SSL) 前端的融合」。由於不同的 SSL 模型(如 WavLM、HuBERT、XLSR 等)對聲學特徵的偏好與注意力機制各異,若能設計有效的聚合網路將這多種前端特徵截長補短地結合,將有望更全面地覆蓋並捕捉各式未知的深偽生成痕跡。